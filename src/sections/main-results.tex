We show that for every odd power $P$ and non-negative integer $N$ it is always true that

\begin{align*}
(N+1)
    ^P - N^P - 1 \equiv 0 \bmod{6}
\end{align*}

To prove the statement we have to show that for every non-negative integer $N$ and odd power $P$

\begin{align*}
    (N+1)^P - N^P - 1
\end{align*}

is always divisible by both 2 and 3.
For every non-negative integer $N$, either $N$ is even, and $N+1$ is odd, and vise versa.
It means that $(N+1)^P - N^P$ is always a positive odd number.
Therefore,
\begin{align*}
    (N+1)^P - N^P - 1 \equiv 0 \bmod{2}
\end{align*}

Let be $F(N) = (N+1)^P - N^P - 1$, then we want to show that for every non-negative $N$

\begin{align*}
    F(N) \equiv 0 \bmod{3}
\end{align*}

Since every integer $N$ is congruent to either 0, 1, or 2 modulo 3, we check each case.
Consider the case $N \equiv 0 \bmod 3$
\begin{align*}
    N &\equiv 0 \bmod 3 \\
    N+1 &\equiv 1 \bmod 3 \\
    N^p &\equiv 0^p = 0 \bmod 3 \\
    (N+1)^p &\equiv 1^p = 1 \bmod 3
\end{align*}
Therefore,
\begin{align*}
    f(N) = (N+1)^p - N^p - 1 \equiv 1 - 0 - 1 = 0 \bmod 3
\end{align*}

Consider the case $N \equiv 1 \bmod 3$
\begin{align*}
    N &\equiv 1 \bmod 3 \\
    N+1 &\equiv 2 \bmod 3 \\
    N^p &\equiv 1^p = 1 \bmod 3 \\
    (N+1)^p &\equiv 2^p \bmod 3
\end{align*}

Since $p$ is odd, $2^p \equiv 2 \bmod 3$, so
\begin{align*}
    f(N) \equiv 2 - 1 - 1 = 0 \bmod 3
\end{align*}

Consider the case  $N \equiv 2 \bmod 3$
\begin{align*}
    N &\equiv 2 \bmod 3 \\
    N+1 &\equiv 0 \bmod 3 \\
    N^p &\equiv 2^p \equiv 2 \bmod 3 \\
    (N+1)^p &\equiv 0^p = 0 \bmod 3 \\
    f(N) &\equiv 0 - 2 - 1 = -3 \equiv 0 \bmod 3
\end{align*}
